\documentclass[addpoints,12pt]{exam}

\makeatletter
\expandafter\providecommand\expandafter*\csname ver@framed.sty\endcsname
{2003/07/21 v0.8a Simulated by exam}
\makeatother

\usepackage{xcolor}
\usepackage{minted}
\usepackage[utf8]{inputenc}
\usepackage{tikz}
\usepackage{caption}
\usepackage{gensymb}
\usepackage{lmodern}
\usepackage{multirow}
\usepackage{booktabs}
\usepackage{array}
\usepackage{adjustbox}
\usepackage{upquote}
\usepackage{amsmath}
\usepackage[hidelinks]{hyperref}
\usetikzlibrary{mindmap,shadows, shapes, arrows, positioning}

\tikzstyle{rect} = [rectangle, fill=ProcessBlue, text width=4.5em, text centered, minimum height=4em, rounded corners]
\tikzstyle{line} = [draw, ->, very thick]
\tikzstyle{oval} = [ellipse, fill=SeaGreen, text width=5em, text centered]

\newcolumntype{x}[1]{>{\centering\arraybackslash\hspace{0pt}}p{#1}}

\renewcommand{\refname}{\selectfont\normalsize References} 
\pagestyle{headandfoot}

\header{\textbf{Problem Sheet: Computational complexity}}{}{Theory of Algorithms}
\footer{}{Page \thepage\ of \numpages}{}
\marksnotpoints
\pointsinrightmargin

\begin{coverpages}
\end{coverpages}

\printanswers

\begin{document}

\noindent
This problem sheet is about computational complexity~\cite{sipserbook}.

\begin{questions}

\question
Determine the number of comparisons made by Bubble sort on the following inputs.
\begin{parts}
  \part $[3,2,1]$
  \part $[4,3,2,1]$
  \part $[5,4,3,2,1]$
  \part $[6,5,4,3,2,1]$
  \part $[20,19,18,\ldots,3,2,1]$
  \part $[3,4,5,2,1]$
  \part $[4,5,1,2,3]$
\end{parts}
\begin{solution}
  \begin{parts}
    \part $2 + 1 = 3$
    \part $3 + 2 + 1 = 6$
    \part $4 + 3 + 2 + 1 = 10$
    \part $5 + 4 + 3 + 2 + 1 = 15$
    \part $6 + 4 + 3 + 2 + 1 = 21$
    \part $(20 \times 19) \div 2 = 190$
    \part $4 + 3 + 2 + 1 = 10$
    \part $5 + 4 + 3 = 12$, using the fact that if no swaps are made in one pass then the list is sorted.
  \end{parts}
\end{solution}


\question
Classify the following as polynomial, exponential, or logarithmic expressions.
  \begin{parts}
    \part $3n + 1$
    \part $n^2 + 2n +1$
    \part $log_b (a)$
    \part $10^n$
    \part $2^n + n^2$
    \part $nlog_n$
    \part $n^n$
  \end{parts}
\begin{solution}
  \begin{parts}
    \part Polynomial (linear).
    \part Polynomial.
    \part Logarithmic.
    \part Exponential.
    \part Ambiguous.
    \part Ambiguous.
    \part Ambiguous.
    \part Exponential.
  \end{parts}
\end{solution}


\question
Explain what the P computational complexity class is, and give an example of a problem known to be in P.


\question
Explain what PRIMES is.
\begin{solution}
  \[ PRIMES = \{2, 3, 5, 7, 11, \ldots \} \]
  PRIMES is the set of all prime numbers, and is a subset of the natural numbers.
\end{solution}


\question
Describe two different algorithms the check if a number is a prime.
The algorithms should accept a single positive integer as input, and output true if the number is prime and false otherwise.


\question
Determine which of the following are in PRIMES (without Google).
\begin{parts}
  \part 2
  \part 3
  \part 4
  \part 10
  \part 11
  \part 13,109
  \part 100,827
  \part 102,203
\end{parts}
\begin{solution}
  \begin{parts}
    \part Yes
    \part Yes
    \part No
    \part No
    \part Yes
    \part Yes
    \part No
    \part Yes
  \end{parts}
\end{solution}


\question
Explain what a decision problem is, and how decision problems relate to Turing machines.
\begin{solution}
As in notes.
\end{solution}


\question
Explain the decision problem related to PRIMES.


\question
Explain the concept of complexity in terms of Turing machines.

\question
Explain why if we can solve decision problem $A$ in polynomial time, and we can convert decision problem $B$ to problem $A$ in polynomial time, then we can solve problem $B$ in polynomial time too.


\end{questions}


\bibliographystyle{plain}
\bibliography{bibliography}
\end{document}
