%!TEX root = problems.tex

% \printanswers

\noindent
This problem sheet is about computational complexity~\cite{sipserbook}.

\begin{questions}

\question
Classify the following as polynomial, exponential, or logarithmic expressions.
  \begin{parts}
    \part $3n + 1$
    \part $n^2 + 2n +1$
    \part $log_b (a)$
    \part $10^n$
    \part $2^n + n^2$
    \part $nlog_n$
    \part $n^n$
  \end{parts}
\begin{solution}
  \begin{parts}
    \part Polynomial (linear).
    \part Polynomial.
    \part Logarithmic.
    \part Exponential.
    \part Ambiguous.
    \part Ambiguous.
    \part Ambiguous.
    \part Exponential.
  \end{parts}
\end{solution}

\question
Explain what decision problems are, and how they relate to Turing machines.
\begin{solution}
As in notes.
\end{solution}

\question
Explain what the P computational complexity class is, and give an example of a problem in P.



\question
Explain the terms conjunctive normal form and disjunctive normal form.
\begin{solution}
As in notes.
\end{solution}


\question
Convert the following expressions to Conjunctive Normal Form.
\begin{parts}
  \part $a \vee b$
  \part $a \wedge b$
  \part $((a \wedge b) \vee ( \neg b \wedge c)) \vee  \neg d$
  \part $(a  \wedge   b) \vee (c  \wedge  d)$
  \part $(a \vee b)  \wedge  (c \vee d)$
\end{parts}
\begin{solution}
\begin{parts}
  \part $a \vee b$
  \part $a \wedge b$
  \part $(a \vee \neg b \vee \neg d) \wedge (b \vee c \vee \neg d)$
  \part $(a\vee c) \wedge (a\vee d) \wedge (b\vee c) \wedge (b\vee d)$
  \part $(a \vee b)  \wedge  (c \vee d)$
\end{parts}
\end{solution}

\question
Convert the following expressions to Disjunctive Normal Form.
\begin{parts}
  \part $a \vee b$
  \part $a \wedge b$
  \part $((a \wedge b) \vee ( \neg b \wedge c)) \vee  \neg d$
  \part $(a \wedge b) \vee (c \wedge d)$
  \part $(a \vee b)  \wedge  (c \vee d)$
\end{parts}
\begin{solution}
\begin{parts}
  \part $a \vee b$
  \part $a \wedge b$
  \part $(a \wedge b) \vee (\neg b \wedge c) \vee \neg d$
  \part $(a  \wedge  b) \vee (c  \wedge  d)$
  \part $(a \wedge c)\vee(a \wedge d)\vee(b \wedge c)\vee(b \wedge d)$
\end{parts}
\end{solution}


\question
Determine if there is a setting of the variables in the following expression that makes the evaluation of the expression true.
\begin{parts}
  \part $a \vee b$
  \part $a \wedge b$
  \part $((a \wedge b) \vee ( \neg b \wedge c)) \vee  \neg d$
  \part $(a  \wedge   b) \vee (c  \wedge  d)$
  \part $(a \vee b)  \wedge  (c \vee d)$
\end{parts}

\begin{solution}
\begin{parts}
  \part $(a,b) = (1,1)$
  \part $(a,b) = (1,1)$
  \part $(a,b,c,d) = (1,1,1,0)$
  \part $(a,b,c,d) = (1,1,1,0)$
  \part $(a,b,c,d) = (1,1,1,0)$
\end{parts}
\end{solution}


\question
Explain how to prove that a problem is NP-complete.
\begin{solution}
As in notes.
\end{solution}


\question
Prove that 3-SAT is NP-complete. You may assume that SAT is NP-complete.
\begin{solution}
As in notes.
\end{solution}


\end{questions}
